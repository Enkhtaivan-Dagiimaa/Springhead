% 2.2.SetInstallDirectory.tex
%	Last update: 2021/05/17 F.Kanehori
%\newpage
\subsection{ライブラリとヘッダファイルのインストール先を指定する}
\label{subsec:SetInstallDirectory}
\parindent=0pt

ライブラリファイルとヘッダファイルのインストール先を設定する場合は、
ファイル\Path{\CMakeConf{}}の内容を次のように編集します。

\begin{narrow}[s]
	\Vskip{-.2\baselineskip}\thinrule{\linewidth}\\
	変数の設定はGUIからでも行なえます。
	ただし GUIでの設定より\Path{CMakeConf.txt}の設定の方が優先されます。
	また、GUIで\tt{CMAKE\_INSTALL\_PREFIX}を空に設定した場合、
	または \Path{CMakeConf.txt}で \tt{DO\_NOT\_GENERATE\_INSTALL\_TARGET}
	を指定した場合には、
	インストールを行なうためのターゲットルールは生成されません。
	なお、\Path{CMakeConf.txt}で\tt{CMAKE\_INSTALL\_PREFIX}に空文字列を
	設定した場合には、GUIの場合とは異なり、デフォルトのインストールパス
 	(Windowsでは\Path{C:/Program Files}、unixでは\Path{/usr/local})が
	使用されます。
	\Vskip{-.2\baselineskip}\thinrule{\linewidth}\\
	\Vskip{.3\baselineskip}
\end{narrow}

変数\tt{CMAKE\_INSTALL\_PREFIX}にインストール先を絶対パスで設定します。
この設定をすることにより、\tt{find\_package(Springhaed, REQUIRED)}として
\SprLib を簡単に導入することができるようになります。
\CmndLine{%
	\cite{set(CMAKE\_INSTALL\_PREFIX ".../where/to/install")} \\
	\cite{\hspace{20pt}\#\hspace{40pt}{(\rm{use absolute path)}}}\\
}{chap-2-2-set-install-prefix.eps}{Set install prefix}
\medskip
実際にファイルがインストールされるディレクトリは
\Vskip{-.3\baselineskip}
\def\IP{\$\{CMAKE\_INSTALL\_PREFIX\}}
\begin{narrow}[s]
	\begin{tabular}{lcl}
		\tt{config files} & → & \tt{\IP/Springhead} \\
		\tt{header files} & → & \tt{\IP/Springhead/include} \\
		\tt{library file} & → & \tt{\IP/Springhead/lib} \\
	\end{tabular}
\end{narrow}
\Vskip{.2\baselineskip}
です。
これらのうち、ヘッダファイルとライブラリファイルのインストール先は
次の変数を設定することで変更が可能です。
絶対パスまたは\tt{\,\IP\,}からの相対パスで指定してください。
\Vskip{-.3\baselineskip}
\CmndLine{%
	\cite{set(SPR\_HEADERS\_INSTALL\_DIR "header files のインストール先")} \\
	\cite{set(SPR\_LIBRARY\_INSTALL\_DIR "library file のインストール先")} \\
}{chap-2-2-set-install-dir.eps}{Set install directory}

\bigskip
% end: 2.2.SetInstallDirectory.tex
