% 2.8.VersionContrilSystem.tex
%	Last update: 2022/02/02 F.Kanehori

%\newpage
\subsection{VersionControlSystemクラス}
\label{subsec:VersionControlSystem}

\begin{Description}{スクリプトファイル}
	\Path{<topdir>/core/test/bin/VersionControlSystem.py}
\end{Description}

\medskip
\begin{Description}{クラス生成}
	\cmnd{VersionControlSystem(url, wrkdir, verbose)}
	\begin{Args}
	  \Item[t]{url}{サーバのurl}
	  \Item[t]{wrkdir}{処理を行なうディレクトリ(Git管理化にあること)}
	\end{Args}
\end{Description}

\medskip
\begin{Description}{クラスの機能}
	バージョンコントロールシステムから情報を取得するためのラッパクラス。
	現在利用できるのはGitHubシステムのみ
	(クラス名は以前処理対象としていたsubversionシステムに由来する)。
\end{Description}

\medskip
メソッドの機能

\skip{-0.5}
\begin{enumerate}
  \item	\Func{revision}\\
	サーバに次のコマンドを送って最新の\tt{commit-id}を取得する。
	\begin{narrow}
		\cmnd{git log --abbrev-commit --oneline --max-count=1}
	\end{narrow}
	返された情報の最初のフィールドが \tt{commit-id(短形式)} なので、
	その部分を切り出して返す。
	
  \item	\Func{revision\_info}\\
	サーバに \cmnd{git log} コマンドを送る。
	次の形式のログ情報が複数返されるので、以下の方法で必要な情報を収集する。
	\begin{narrow}
		\tt{commit \it{長形式commit-id}}\\
		\tt{Merge: ……}\\
		\tt{Author: ……}\\
		\tt{Date: \it{日付}}\\
		\tt{……}
	\end{narrow}
	\begin{enumerate}
	  \item	\label{9.0.enum.1}%
		commit行が現れたら新しいcommitの開始と認識して\tt{commit-id}を記録する。
		次のcommit行が現れるまではすべてこのcommitに属するものとする。
	  \item	Date行が現れたら日付・時刻を取り出し、
		\tt{"YYYY-mmdd,hh:mm:ss"}形式で記録する。
	  \item	(\ref{9.0.enum.1})で記録したcommit-idが
		指定されたcommit-idまたは\tt{HEAD}であるならばこれで終了。
		収集した情報を返す。
		さもなければ以上を繰り返す。
	\end{enumerate}
	
  \item	\Func{get\_file\_content}\\
	サーバにgit \cmnd{show commit-d:path} を送ってファイルの内容を取得する。
\end{enumerate}

% end: 2.8.VersionControlSystem.tex
