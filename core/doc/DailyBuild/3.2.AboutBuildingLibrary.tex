% 3.3.AboutBuildingLibrary.tex
%	Last update: 2022/02/10 F.Kanehori

%\newpage
\subsection{Springheadライブラリビルドの流れ}
\label{subsec:AboutBuildingLibrary}

% -----------------------------------------------------------------------
\subsubsection{RunSwig}
\label{subsubsec:RunSwig}
% -----------------------------------------------------------------------
\begin{narrow}[s]
	Visual Studio でのビルドにあたっては、
	セットアップの実行は不要である
	(Visual Studio は nmake などのパスを自前で組み込んでくれるので
	セットアップ情報は必要ない)。
	しかし、各スクリプトをcmake版でも使えるようにする必要から、
	各スクリプトの中でセットアップファイルの有無を確認するコードが
	組み込まれている(\tt{if SetupExists}など)。
	これらは無視すること。
\end{narrow}
\medskip
ディレクトリは\Path{<topdir>/core/src/RunSwig}。\\
RunSwigはmakeプロジェクトで、実行されるコマンドは次のとおり。
\begin{narrow}
	build → \cmnd{nmake /NOLOGO -f Makefile.win}\\
  	clean → \cmnd{..\do\_python.bat RunSwig\_clean.py --RunSwig}
\end{narrow}
			
\medskip
処理の概要

\begin{enumerate}
  \item	build (\Path{Makefile.win})\\
	ターゲット設定
	\skip{-0.3}
	\begin{narrow}
	\begin{tabular}{l@{\Hskip{10pt}}l}
	    \tt{all}:	  & \tt{PreWork RunSwig}\\
	    \tt{PreWork}: & \tt{ClosedSrcCtl \$(SCILABSTUBHDR) \$(FWOLDSPRSTUB)}
	\end{tabular}
	\end{narrow}
	\begin{Description}{ターゲット \tt{ClosedSrcCtl}}
		スクリプト\Path{.\BS CheckClosedSrc.py}を実行して、
		closed sourceを使用してビルドするか否かを制御する。\\
		\Path{<topdir>\BS core\BS include}に
		\Path{SprUseClosedSrcOrNot.h}が存在しないときに限り、
		次の内容を
		\begin{narrow}
			\Path{<topdir>\BS core\BS include\BS SprUseClosedSrcOrNot.h}
		\end{narrow}
		として書き出す
		(\Path{<topdir>/core/test/bin}にテンプレートファイルが
		あるのでそれをコピーする)。
		\begin{narrow}{\small{%
			\begin{tabular}{l@{ → }l}
			    \Path{<topdir>\BS closed}が存在する
				& \tt{"\#define USE\_CLOSED\_SRC"}\\
  			    \Path{<topdir>\BS closed}が存在しない
				& \tt{"\#undef USE\_CLOSED\_SRC"}
			\end{tabular}
		}}\end{narrow}
		\skip{.5}
		\Path{SprUseClosedSrcOrNot.h}は、
		\Path{SprDefs.h}から\tt{\#include}される。
	\end{Description}
	\begin{Description}{ターゲット \tt{\$(SCILABSTUBHDR)}}
		スクリプト\Path{..\BS Foundation\BS ScilabSwig.py}を実行する。\\
		Scilabモジュールについて、
		\cmnd{swig}を実行するためのmakefileを作成し、
		\cmnd{make}を実行する\\
		(\Path{Scilab.i} → 
		\Path{..\BS ..\BS include\BS Scilab\BS ScilabStub.hpp})。\\
		ただし、\Path{ScilabStub.hpp}は、\Path{dll.cpp}を
		移植しないと作成できないので、
		\cmnd{make}は実行できない(\cmnd{make clean}も実行できない)。\\
		実際には、\Path{Scilab.i}及び\Path{ScilabStub.hpp}は
		作成済みのものがコミットされているので、
		このターゲット自体が実行されない。
	\end{Description}
	\begin{Description}{ターゲット \tt{\$(FWOLDSPRSTUB)}}
		スクリプト\Path{RunSwigFramework.py}を実行する。\\
		作業領域として\Path{.\BS tmp\BS}を使用する。\\
		このターゲットは次のファイルを作成する。\\
		\begin{tabular}{@{\Hskip{10pt}}ll}
		    \Path{FWOldSpringhead.i} & {\small (作業領域のみ)}\\
		    \Path{FWOldSpringheadStub.makefile}
					& {\small (作業領域のみ)}\\
		    \Path{...\BS include\BS SprFWOldSpringheadDecl.hpp}
					& {\small (作業領域からコピー)}\\
		    \Path{...\BS src\BS FWOldSpringheadDecl.hpp}
					& {\small (作業領域からコピー)}\\
		    \Path{...\BS src\BS FWOldSpringheadStub.cpp}
					& {\small (作業領域からコピー)}\\
		\end{tabular}
	\end{Description}
	\begin{Description}{ターゲット \tt{RunSwig}}
		スクリプト\Path{do\_swigall.py}を実行する。\\
		この中で\Path{.\BS RunSwig\_method.py}がインクルードされる。\\
		処理対象とするプロジェクト(Collision, Creature, ...)の
		各々について、次の処理を行なう(Baseは除外)。
		\begin{narrow}[20pt]
		    処理対象プロジェクト\tt{<proj>}のディレクトリに移動して
		    \Func{RunSwig}を呼び出す。これにより
		    \begin{itemize}
			\item \tt{"<proj>.i"}の生成
				(この時点でのヘッダファイル情報を収集する)
			\item \tt{"<proj>Stub.makefile"}の生成
			\item \cmnd{make "<proj>Stub.makefile"}の実行
		    \end{itemize}
  		    が行なわれ、条件が満たされれば\cmnd{swig}が実行される。
		\end{narrow}
	\end{Description}

  \item	clean (\Path{..\BS do\_python.bat RunSwig\_clean.py})\\
	Visual Studioでソリューションのクリーンを実行しても、
	RunSwigが作成したファイルは削除されずに残ってしまう。
	このため、これらのファイルを削除するためのスクリプトを用意して
	RunSwigのクリーンから呼び出している。

	\Path{RunSwig\_clean.py}には2つのオプションがある。\\
	\def\wid{240pt}
	\begin{tabular}{@{\Hskip{10pt}}ll}
	    \tt{-C, --cmake} & \FCol{\wid}{%
		\tt{cmake}が作成したファイル
			(サイズ0のダミーファイル)を削除する。}\\
	    \tt{-R, --RunSwig} & \FCol{\wid}{%
		RunSwigが作成したファイル(次のもの)を削除する。\\
		{\small
	  	\tt{<proj>Stub.makefile, <proj>.i, <proj>Stub.cpp,}\\
	  	\tt{<proj>Decl.hpp, swig\_spr.log,}\\
	  	\tt{.../include/FWOldSpringhead,}\\
	  	\tt{.../src/Framework/FWOldSpringheadStub.cpp,}\\
	  	\tt{.../src/Framework/FWOldSpringheadDecl.hpp,}}}
	\end{tabular}\\
	\skip{.2}
	ただし、\tt{-C}オプションは現在は使用していない
	(\file{CMakeLists.txt}で解決済み)。
\end{enumerate}

% -----------------------------------------------------------------------
\subsubsection{Springhead}
\label{subsubsec:Springhead}
% -----------------------------------------------------------------------
ディレクトリは\Path{<topdir>\BS core\BS src"}。\\
Springheadはmakeプロジェクトで、実行されるコマンドは
\begin{narrow}
	build → \cmnd{SpringheadLib.bat x64 16.0x64} など\\
  	clean → \cmnd{del ..\BS ..\BS generated\BS lib\BS win64\BS 
			pringhead16.0x64.lib} など
\end{narrow}
である。
	
\begin{Description}{build}
	スクリプト\Path{.\BS Springhead.bat}を実行する。\\
	\begin{tabular}{@{\Hskip{10pt}}ll}
	    第1引数 & platform (\tt{x86}/\tt{win32}または\tt{x64}/\tt{win64})\\
	    第2引数 & 構成等識別情報 (\tt{[VS-version]+[configuration]+[platform]})
	\end{tabular}

	各プロジェクトのディレクトリにある\Path{.lib}を集めて
	\Path{Springhead<ext>.lib}を作成する。
	ここで\tt{<ext>}は構成等識別情報 (第2引数)。
\end{Description}

% end: 3.3.AboutBuildingLibrary.tex
