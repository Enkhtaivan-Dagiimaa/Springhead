% 3.4.VisualStudioNewVersion.tex
%	Last update: 2022/02/07 F.Kanehori

%\newpage
\subsection{Visual Studio 新バージョンへの対応方法}
\label{subsec:VisualStudioNewVersion}

これまでの説明の中でも触れてきたが、
Visual Studioの新しいバージョンに対応するためには
いくつかの追加・修正が必要となる。
ここで、それらをまとめておく。

\medskip
\begin{Description}[b]{Visual Studioのバージョン情報}
	\Class{VisualStudio} (\Path{VisualStudio.py})にバージョン情報を追加する。
	\skip{-1.0}
	\begin{narrow}[40pt]
	\begin{Table}[r][80pt]{l@{ … }l}
	  \Item{\Func{\_\_get\_vsinfo}}{条件節を追加}
	  \Item{\Func{\_\_get\_vs\_path}}{%
		\cmnd{vswhere}でパスを見つけられないときは
		ここで対処しないといけない。}
	\end{Table}
	\end{narrow}
	\RefRef{subsec}{VisualStudio}
\end{Description}

\medskip
\begin{Description}[b]{cmakeのgenerator}
	\Class{CMake} (\Path{CMake.py})のクラス定数\var{GENERATOR}に追加する。\\
	前項と重複しているが、こちらも追加する必要がある。\\
	現在定義されているのは
	\begin{narrow}[40pt]
	\begin{tabular}{l@{\Hskip{5pt}$\Leftrightarrow$\Hskip{5pt}}l}\hline
		2015 & \tt{"Visual Studio 14 2015" -A x64}\\
		2017 & \tt{"Visual Studio 15 2017" -A x64}\\
		2019 & \tt{"Visual Studio 16 2019" -A x64}\\\hline
	\end{tabular}
	\end{narrow}
	である。\\
	左側の数字は、内部メソッド\Func{\_\_find\_generator}で判定した
	Visual Studioのバージョン数字。\\
	\RefRef{subsec}{Traverse}
\end{Description}

% end: 3.4.VisualStudioNewVersion.tex
