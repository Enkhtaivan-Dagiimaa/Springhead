% 2.7.RevisionInfo.tex
%	Last update: 2022/02/02 F.Kanehori

%\newpage
\subsection{RevisionInfoスクリプト}
\label{subsec:RevisionInfo}

\begin{Description}{スクリプトファイル}
	\Path{<topdir>/core/test/bin/RevisionInfo.py}
\end{Description}

\medskip
\begin{Description}{起動形式}
	\cmnd{python RevisionInfo.py [options] revision}
	\begin{Args}
	  \Item[t]{revision}
		{\tt{-f}オプション指定時には\tt{commit-id}\\
		それ以外の場合は\tt{all}を指定してすべてのcommitを対象とする}
	\end{Args}
	\begin{Opts}
	  \Item[t]{-s}{Github (GitHub.com) を使用する}
	  \Item[t]{-R}{Gitlab (git.haselab.net) を使用する}
	  \Item[t]{-f fname}{リポジトリから取り出すファイル名}
	  \Item[t]{-u}{unixで使用する場合に指定}
	\end{Opts}
\end{Description}

\def\HD#1{\hfill\bf{#1}\hfill}

\medskip
動作の概要
\begin{enumerate}
  \item	\tt{-f}オプション指定時\\
	\Func{info}で\tt{commit-id}情報を収集したら、
	\Func{contents}でそれらの情報を出力する。

  \item	それ以外\\
	\Func{info}で\tt{commi-id}情報を出力する。\\
	出力形式:
	\begin{narrow}
		\tt{"commit-id(短形式),commit-id(長形式),日付"}
	\end{narrow}
\end{enumerate}

\medskip
関数の機能
\begin{enumerate}
  \item	\Func{info}\\
	指定されたcommit-idに関して、
	\Func{VersionControlSystem.revision\_info}を用いて情報
	\begin{narrow}
		\tt{"commit-id(短形式),commit-id(長形式),日付"}
	\end{narrow}
	を取得する。\\
	引数\tt{out}が指定されていたら、これらの情報を出力する。

  \item	\Func{contents}\\
	\Func{VersionControlSystem.get\_file\_content}を用いて、
	指定\tt{commit-id}版におけるファイル\tt{fname}
	およびファイル\file{Springhead.commit.id}
	(このファイルにはテストが実行された時の\tt{commit-id}
	および日付の履歴が記録されている)の内容を読み出す。
	不要な情報でなければ(いくつか紛れ込んでいるのでそれらは無視する必要がある)
	\tt{commit-id}と日付とともにファイルの内容を出力する。
\end{enumerate}

% end: 2.7.RevisionInfo.tex
