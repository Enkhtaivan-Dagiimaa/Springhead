% 2.6.VisualStudio.tex
%	Last update: 2022/02/02 F.Kanehori

%\newpage
\subsection{VisualStudioクラス}
\label{subsec:VisualStudio}

\begin{Description}{スクリプトファイル}
	\Path{<topdir>/core/test/bin/VisualStudio.py}
\end{Description}

\medskip
\begin{Description}{クラス生成}
	\cmnd{VisualStudio(toolset, verbose)}
	\begin{Args}
	  \Item[t]{toolset}{Visual Studio toolset ID}
	\end{Args}
\end{Description}

\medskip
\begin{Description}{クラスの機能}
	コマンドラインからVisual Studioを起動するためのラッパクラス。
\end{Description}

\medskip
メソッドの機能

\begin{enumerate}
  \item	\Func{solution}\\
	Solution fileの名前とパスを記録する。
	引数\tt{name}はVisual Studioのバージョンを含んでいること。
	
  \item	\Func{build}\\
	前処理をした後、
	内部メソッド\Func{\_\_build}(後述)を呼び出してビルドを実行する。
	
  \item	\Func{\_\_get\_vsinfo} ({\small 内部メソッド})\\
	与えられた引数を手掛かりとして、Visual Studioのバージョン情報を確定する。
	\begin{narrow}
	\begin{tabular}{lll}
	    \tt{pts} & platform toolset version	& e.g. v142\\
	    \tt{vsv} & Visual Studio vereion	& e.g. 16.0\\
	    \tt{vsn} & Visual Studio name	& e.g. Visual Studio 2019\\
	\end{tabular}
	\skip{.5}
	\end{narrow}

	\tt{vsv}が“バージョン情報”に相当する。\tt{pts}と\tt{vsn}はメッセージ用。\\
	\Color{red}{%
		\bf{新しいVisual Studioを導入する場合は、
		このメソッドに情報を追加する必要がある。}
	}

  \item	\Func{\_\_get\_vs\_path} ({\small 内部メソッド})\\
	\cmnd{devenv}のパスを求める。
	現在DESKTOP-KD3C7HS (demo)マシンにはVisual Studio version 15.0以外は
	インストールされていない(16.0に対応しているのはkanehori-PCだけ)。\\
	\Color{red}{\bf{%
	新しいVisual Studioを導入する場合は、
	このメソッドにも情報を追加する必要がある。}}

\end{enumerate}

% end: 2.6.VisualStudio.tex
