% 5.3.1.order.tex
%	Last update: 2022/02/14 F.Kanehori
%\newpage
\subsubsection{order.pl}
\label{subsubsec:order}

\medskip
Visual Studio (devenv) が出力したログを、モジュール単位かつスレッド順に並べ替える。
実際の処理はライブラリ(\tt{dailybuild\_lib.pm})にある
\tt{read\_log()}が行なう(\RefRef{subsubsec}{dailybuildlib})。

\medskip
\begin{Description}{起動形式}
	\cmnd{perl order.pl [options] [infile]}
	\begin{Opts}[b]
	  \Item[t]{-o outfile}{出力ファイル名 (stdout\Note)}
	  \Item[t]{-p patt\_id}{モジュール識別パターン番号}
	\end{Opts}
	\begin{Args}[b]
	  \Item[t]{infile}{入力ファイル名 (stdin\Note)}
	\end{Args})
\end{Description}

処理の概要
\begin{enumerate}
  \item	\tt{dailybuild\_lib.pm}のライブラリ関数\Func{read\_log}を用いて
	入力ファイルの読み込み処理を行なう。
	返されるのは(\plHshR{modules}, \plAryR{modules}, \plAryR{lines})の3つ組である。
	\begin{narrow}[10pt]\begin{tabular}{ll}
	    \plHsh{modules}
		& モジュール名をキーとした連想配列(要素は\plHsh{threads})\\
	    \plHsh{threads}
		& スレッド番号をキーとした連想配列(要素は入力データ行の配列)\\
	    \plAry{modules}
		& モジュール名の配列(出現順)
	\end{tabular}\end{narrow}

  \item	入力データ行を\plHsh{threads}からモジュール順、スレッド順に取り出して、
	\tt{outfile}(指定がなければ標準出力)に書き出す。

\end{enumerate}

% end: 5.3.1.order.tex
