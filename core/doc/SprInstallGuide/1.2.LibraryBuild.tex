% 1.2.LibraryBuild.tex
%	Last update: 2021/03/15 F.Kanehori
%\newpage
\subsection{Springhead Library のビルド}
\label{subsec:Library_Build}
\parindent=0pt

\SprLib のビルド方法について説明します。
ただし、サンプルプログラムをビルドする場合に限りこの節での作業は不要です。

\bigskip
ディレクトリ\SprTop{\core\src}へ移動します。

\bigskip
複数のソリューションファイルがあります。
使用するVisual Studioのバージョンに合ったものを使用してください。

\Important{%
	\SlnFile{Springhead}{nn.n}の \it{nn.n} の部分がバージョンを表します。\\
	\Vskip{-.8\baselineskip}
	\hspace{20pt}\tt{14.0}\ \ ---\ \ 2015用\\
	\hspace{20pt}\tt{15.0}\ \ ---\ \ 2017用\\
	\hspace{20pt}\tt{16.0}\ \ ---\ \ 2019用
}

\bigskip
ソリューションファイルをVisual Studioで起動したら、
Springheadを``スタートアッププロジェクト''に指定してビルドします。
ライブラリファイルは、\\
\hspace{20pt}\Path{C:\Springhead\generated\lib\win64} \\
または\\
\hspace{20pt}\Path{C:\Springhead\generated\lib\win32} \\
のいずれかに生成されます。

\medskip
ライブラリファイル名は次のようになります。
\begin{center}\begin{tabular}{lll}\hline
	構成名 & ライブラリファイル名 & ビルド設定 \\\hline
	\tt{Release} & \tt{Springhead\#\#.lib}  & multithread, DLL \\
	\tt{Debug}   & \tt{Springhead\#\#D.lib} & multithread, Debug, DLL \\
	\tt{Trace}   & \tt{Springhead\#\#T.lib} & multithread, Debug, DLL \\\hline
	\multicolumn{3}{l}{\footnotesize{\vbox{\vbox to 1mm{}
		\hbox{・ \tt{\#\#}は、Visual Studioバージョン及び
			プラットフォームを表す\url{Win32}又は\url{x64}との組み合}
		\hbox{\phantom{・ }わせとなります(\Path{15.0x64}など).}
		\hbox{・ \tt{Trace}構成とは,
			フレームポインタ情報付き\url{Release}構成のことです.}}}}
\end{tabular}\end{center}

\bigskip
% end 1.2.LibraryBuild.tex
