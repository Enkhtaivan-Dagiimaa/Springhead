% 1.2.WhenUsedCMake.tex
%	Last update: 2019/07/10 F.Kanehori
%\newpage
\subsection{CMakeを使用した場合}
\label{subsec:WhenUsedCMake}

\noindent
\cmake は、Visual Studioやmakeのようにソースコードのビルドやデバッグの制御を
することを目的としたツールではなく、
ビルドの自動化を図るために\CMakeLists というパラメータファイルから
solution/project file (Visual Studio)やMakefile (make)を
自動的に生成するツールです
(configureに近いものと考えていただいてよいでしょう)。

\cmake はout of source (out of place)でのビルドに対応しています。
これはソースツリーの外側にビルドツリーを生成する機能で、
\begin{itemize}
  \item	複数のビルドツリー(互いに干渉しない)を作成することが可能である。
  \item	ビルドツリーが削除されてもソースツリーに影響が及ばない。
\end{itemize}
という特徴があります。
我々は\cmake をout of sourceで使用します。

\medskip
ソースツリーおよびビルドツリーは次のようになるでしょう。

\medskip
\begin{narrow}\begin{figure}
    \begin{narrow}\begin{minipage}{\textwidth}
	\dirtree{%
		.1 \hspace{-10mm}.../Springhead/core/src/ \Anno{ソースツリー}.
		.2 CMakeLists.txt.
		.2 Base/ \Anno{(*1)}.
		.3 CMakeLists.txt.
		.3 Affine.cpp.
		.3 :.
		.2 Collision/ \Anno{(*2)}.
		.3 :.
		.2 :.
		.2 \build \Anno{ビルドツリー (他の場所でも構わない)}.
		.3 Springhead.sln.
		.3 Base/.
		.4 Base.vcxproj.
		.4 Base.dir/ \Anno{オブジェクトファイル(\tt{.obj})が置かれる}.
		.4 CMakeFiles/.
		.5 generate.stamp \Anno{スタンプファイル}.
		.3 Collision/.
		.3 :.
	}
	\medskip
    \end{minipage}\end{narrow}
    \caption{Springhead Library構成}
    \label{fig:SpringheadLibraryTree}
\end{figure}\end{narrow}

\newpage
\begin{narrow}\begin{figure}
    \begin{narrow}\begin{minipage}{\textwidth}
	\dirtree{%
		.1 \hspace{-10mm}.../Application/ \Anno{ソースツリー}.
		.2 CMakeLists.txt.
		.2 main.cpp.
		.2 Sub/.
		.3 CMakeLists.txt.
		.3 sub.cpp.
		.3 :.
		.2 Base/ \Anno{Spsringheadの(*1)を指すようにする}.
		.2 Collision/ \Anno{Spsringheadの(*2)を指すようにする}.
		.2 :.
		.2 \build \Anno{ビルドツリー (他の場所でも構わない)}.
		.3 Application.sln.
		.3 Base/.
		.4 Base.vcxproj.
		.4 Base.dir/ \Anno{オブジェクトファイル(\tt{.obj})が置かれる}.
		.4 CMakeFiles/.
		.5 generate.stamp \Anno{スタンプファイル}.
		.3 Collision/.
		.3 :.
	}
	\medskip
    \end{minipage}\end{narrow}
    \caption{Application構成}
    \label{fig:ApplicationTree}
\end{figure}\end{narrow}

\medskip
\noindent
先程述べたように\CMakeLists は\cmake を制御するためのパラメータファイルで、
メインディレクトリおよびサブディレクトリのそれぞれに作成します。

% end: 1.2.WhenUsedCMake.tex
