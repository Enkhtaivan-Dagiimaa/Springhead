% 2.1.PreparationForApp.tex
%	Last update: 2020/02/13 F.Kanehori
%newpage
\subsection{ビルドの準備}
\label{subsec:PreparationForApp}
\parindent=0pt

ディレクトリ\AppTop{}に移動してください。

配布されたファイル
\QCMakeTopdir{.dist}、\QCMakeLists{.Dev.dist}、\QCMakeSettings{.Dev.dist}を、
それぞれ
\QCMakeTopdir{}、\CMakeLists{}、\QCMakeSettings{}
という名前でコピーします。

\CmndLine{%
	> chdir C:/Develop/Application\\
	> \it{copy} \CMakeTopdir{.dist} \CMakeTopdir{}\\
	> \it{copy} \CMakeLists{.Dev.dist} \CMakeLists{}\\
	> \it{copy} \CMakeSettings{.Dev.dist} \CMakeSettings{}\\
}{command-2-1-a.eps}{\CMakeLists{}.App}

\medskip
\Important{%
	\tt{.dist}ファイルが誤って変更されてしまうのを防ぐためにも、
	リネームではなくコピーをお願いします。}

\bigskip
\bf{\QCMakeTopdir{}}の編集

\medskip
\SprLib{}をダウンロードしたディレクトリを\QCMakeTopdir{}に設定します。
これは、CMakeにSpringheadのソースツリーの場所を教えるため
(Libraryのソースを\tt{add\_subdirectory}するため)に必要な設定です。

\CmndLine{%
	> \it{edit} \CMakeTopdir{}\\
	\cite{\#set(TOPDIR "\SprTop{}")}\\
	\hspace{20pt}$\downarrow$\\
	\cite{set(TOPDIR "\SprTop{}")}\\
}{command-2-2-a.eps}{\CMakeTopdir{}.App}

\bigskip
\bf{\QCMakeSettings{}}の編集

\medskip
アプリケーションのビルド条件を設定します。 各変数の意味は次のとおりです。

\def\SetRelPath{\tt{RELATIVE \$\{CMAKE\_SOURCE\_DIR\}}}
\def\CMakeSrcDir{\tt{\$\{CMAKE\_SOURCE\_DIR\}}}

\def\HLine{\cline{2-3}}
\def\Width{255pt}
\def\LBox#1{%
	\vbox{\hbox{\tt{#1}}\vfill}}
\def\RBox#1{%
	\parbox[t]{\Width}{%
	\Vskip{-.4\baselineskip}#1\Vskip{-.3\baselineskip}}}
	
\begin{longtable}{p{4.75pt}|p{84.5pt}|p{\Width}|}\HLine
    %&\multicolumn{1}{c|}{変数名} & \multicolumn{1}{c|}{説明} \\\HLine
    %\endfirsthead
    %&\multicolumn{1}{c}{変数名} & \multicolumn{1}{c}{説明} \\\HLine
    %\endhead
    &\tt{ProjectName} & \RBox{プロジェクト名}\\\HLine
    &\tt{OOS\_BLD\_DIR}
	& \RBox{CMakeの作業領域(ディレクトリ)の名前\\
		本ドキュメントで\BldDir としているもの。} \\\HLine
    &\multicolumn{2}{|l|}{\tt{CMAKE\_CONFIGURATION\_TYPES}} \\
	&& \RBox{ビルド構成\\
		unixの場合はここに複数の構成を指定することができません。
		作業ディレクトリを分けることで対処してください
		(\tt{OOS\_BLD\_DIR}参照)。} \\\HLine
    &\tt{LIBTYPE}
	& \RBox{作成するライブラリの種別\\
		Windowsの場合は\tt{STATIC}を指定してください。unixの場合は、
		\tt{STATIC}なら\tt{.a}を\tt{SHARED}なら\tt{.so}を作成します。\\
		(現在\tt{SHARED}は実装されていません)} \\\HLine
    &\tt{SRCS}
	& \RBox{ビルドの対象とするファイル\\
		設定は\tt{set(SRCS …)}または\tt{file(GLOB SRCS …)}とします。
		後者ではワイルドカードが使えます。\\
		\tt{SRCS}の直後に``\tt{RELATIVE <\it{base-dir}>}''を付加すると
		相対パス指定となります。デフォルトは \\
		%\ \ {\footnotesize{\tt{file(GLOB \SetRelPath\ *.cpp *.h)}}} \\
		\hspace{5pt}{\small{\tt{file(GLOB \SetRelPath\ *.cpp *.h)}}} \\
		です。} \\\HLine
    &\tt{EXCLUDE\_SRCS}
	& \RBox{ビルドの対象から外すファイル\\
		\tt{SRCS}でワイルドカードを使用した場合に有用です。
		上の\tt{SRCS}で\tt{RELATIVE}としていないときは
		絶対パスで指定します。} \\\HLine
    &\tt{SPR\_PROJS}
	& \RBox{アプリケーションに組み込む\SprLib のプロジェクト名
		(この中にRunSwigを含めてはいけません)\\
		unixで\Path{libSpringhead.a}をリンクするときは
		\tt{\$\{EMPTY\}}のままとします。} \\\HLine
    &\tt{DEFINE\_MACROS\_ADD}
	& \RBox{追加のコンパイルマクロ指定 (\tt{/D}, \tt{-D}は不要です)} \\\HLine
    &\tt{INCLUDE\_PATHS\_ADD}
	& \RBox{追加のインクルードパス指定 (\tt{/I}, \tt{-I}は不要です) \\
		現在のディレクトリは \CMakeSrcDir で参照できます。} \\\HLine
    &\tt{COMP\_FLAGS\_ADD}
	& \RBox{追加のコンパイラフラグ指定} \\\HLine
    &\tt{LINK\_FLAGS\_ADD}
	& \RBox{追加のリンカフラグ指定} \\\HLine
    &\tt{LIBRARY\_PATHS\_ADD}
	& \RBox{追加のライブラリパス指定 (\tt{-L}は不要です)} \\\HLine
    &\tt{LIBRARY\_NAMES\_ADD}
	& \RBox{追加のライブラリファイル名 (\tt{-l}は不要です)} \\\HLine
    &\multicolumn{2}{|l|}{\tt{EXCLUDE\_LIBRARY\_NAMES}} \\
	&& \RBox{リンクの対象から外すライブラリファイル名\\
		デフォルトで組み込まれてしまうライブラリファイルを
		排除するために指定します。} \\\HLine
    &\multicolumn{2}{|l|}{\tt{DEBUGGER\_WORKING\_DIRECTORY}} \\
	&& \RBox{Visual Studio Debuggerの作業ディレクトリ名\\
		デバッガはこのディレクトリで起動されたように振る舞います。} \\\HLine
    &\multicolumn{2}{|l|}{\tt{DEBUGGER\_COMMAND\_ARGUMENTS}} \\
	&& \RBox{Visual Studio Debuggerに渡すコマンド引数} \\\HLine
\end{longtable}

別途インストールしているパッケージ(boost, glew, freeglut, glui)を使用する場合には、
配布されたファイル\QCMakeConf{.dist}を\QCMakeConf{}という名前でコピーして
必要な編集をします。

\medskip
ビルドの条件(コンパイル / リンク)を変更したいときは、
配布されたファイル\QCMakeOpts{.dist}を\QCMakeOpts{}という名前でコピーして
必要な編集をします。

\medskip
\bf{ファイル\QCMakeLists{}の変更は必要ありません。}

\medskip
以上で準備作業は終了です。

% end: 2.1.PreparationForApp.tex
