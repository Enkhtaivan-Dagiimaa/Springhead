% 1.2.Preparation.tex
%	Last update: 2020/02/13 F.Kanehori
%\newpage
\subsection{ビルドの準備}
\label{subsec:Preparation}
\parindent=0pt

ダウンロードが済んだら、ディレクトリ\SprTop{/core/src}に移動してください。

ライブラリのビルドに関連する配布ファイルは次のものです。

\begin{center}
\begin{tabular}{l@{\ \ ---\ \ }l}\hline
	\tt{\CMakeLists{.dist}} & ライブラリ生成用設定ファイル \\
	\tt{\CMakeSettings{.dist}} & ビルドパラメータ変更用ファイル \\
	\tt{\CMakeOpts{.dist}} & デフォルトビルドパラメータファイル \\
	\tt{\CMakeConf{.dist}} & 外部パッケージ・インストール先設定用ファイル \\\hline
\end{tabular}
\end{center}

\bigskip
配布されたファイル\QCMakeLists{.dist}を\QCMakeLists{}という名前でコピーします。

\CmndLine{%
	> chdir C:/Springhead/core/src\\
	> \it{copy} \CMakeLists{.dist} \CMakeLists{}
}{command-1-2-a.eps}{CMakeLists.txt}

\medskip
配布されたビルド条件で問題なければ、これで準備は終了です。
\KQuote{\ref{subsec:Build} ビルド}へ進んでください。

\bigskip
独自にインストールしたパッケージ boost, glew, freeglut, gluiを使用する場合
およびライブラリファイルとヘッダファイルのインストール先を指定する場合には、
配布されたファイル\QCMakeConf{.dist}を\QCMakeConf{}という名前でコピーして
必要な編集をします。
編集の方法は\QCMakeConf{}に記述されています。

\def\cite#1{\hspace{10pt}\footnotesize{#1}}
\def\somewhere{"C:/\it{somewhere}/\it{appropreate}"}
\CmndLine{%
	> \it{copy} \CMakeConf{.txt} \CMakeConf{}\\
	> \it{edit} \CMakeConf{}\\
	\hspace{20pt}:\\
	\cite{set(CMAKE\_PREFIX\_PATH "C:/somewhre/appropreate")} \\
	\cite{\hspace{20pt}\#\hspace{40pt}{(\rm{use absolute path)}}}\\
	\cite{\hspace{20pt}\#\hspace{40pt}%
		{(\rm{multiple paths must be separated by `newline' or `semicolon')}}}\\
	\cite{)}\\
	\hspace{20pt}:\\
	\cite{set(SPRINGHEAD\_INCLUDE\_PREFIX       "C:/somewhere/appropreate)}\\
	\cite{set(SPRINGHEAD\_LIBRARY\_DIR\_DEBUG   "C:/somewhere/appropreate)}\\
	\cite{set(SPRINGHEAD\_LIBRARY\_DIR\_RELEASE "C:/somewhere/appropreate)}
}{command-1-2-b.eps}{CMakeConf.txt}

\medskip
\begin{narrow}[s][20pt]
	変数\it{variable}に値\it{value}を設定するには
	\tt{set(\it{variable} "\it{value}"}) とします。
	途中に空白やセミコロンを含まない文字列ならば引用符は省略できます。
	また、\tt{\$\{\it{variable}\}}とすると他の変数の値を、
	\tt{\$ENV\{\it{variable}\}}とすると環境変数の値を参照できます。
	文字`\tt{\#}'以降はコメントです。
\end{narrow}

\bigskip
コンパイル及びリンクのオプションはファイル\QCMakeOpts{.dist}に設定されています。
これらのオプションを変更するときは、配布されたファイル\QCMakeOpts{.dist}を
\QCMakeOpts{}という名前でコピーして必要な編集をします。

\CmndLine{%
	> \it{copy} \CMakeOpts{.dist} \CMakeOpts{}\\
	> \it{edit} \CMakeOpts{}\\
	\hspace{20pt}:\\
}{command-1-2-c.eps}{CMakeOpts.txt}

\bigskip
以上で準備作業は終了です。

\bigskip
\thinrule{\linewidth}

\bf{[注意]}

ダウンロード後最初のビルド時にswigのバイナリを生成しますが、
この時使用するプログラム\tt{devenv.exe}がパスに含まれていないと、
RunSwigに失敗して\SprLib が作成できません。
これはVisual Studioをデフォルト以外の場所にインストールしている場合などで起きる現象です。

この現象は、次のようにすると解消できます。

\begin{enumerate}
  \item	``Visual Studio用開発者コマンドプロンプト''を起動する
  \item	コマンド\Path{where devenv.exe}を実行してパスを調べる
  \item	上で調べたパスを環境変数\tt{PATH}に加える
  \item	Visual Studioを再起動する
\end{enumerate}

% end: 1.2.Preparation.tex
