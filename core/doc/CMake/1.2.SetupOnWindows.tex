% 1.2.SetupOnWindows.tex
%	Last update: 2021/05/17 F.Kanehori
%\newpage
\subsection{Windowsでのセットアップ}
\label{subsec:SetupOnWindows}
\parindent=0pt

\subsubsection{初めてのセットアップ}
\label{subsubsec:FirstSetupOnWindows}

ディレクトリ\Path{.../Springhead/core/src}に移動して、
スクリプト\tt{setup.bat}を実行します。

\CmndLine{%
	> chdir .../Springhead/core/src \\
	> setup.bat \\
	-- found python: C:/Python/python.exe \\
	\\
	setup file ("setup.conf") not exists. \\
	\\
	currently available binaries are ... \\
	-- checking python ... found (version: 3.5.4) \\
	-- checking devenv ... selection\_number: None \\
	    found (version: 15.9.28307.222) \\
	-- checking swig ... found (version: 2.0.4) \\
	-- checking cmake ... found (version: 3.18.1) \\
	\\
	check result is ... \\
	-- setup is required (reason: "CMakeLists.txt" does not exist, \\
	setup file "setup.conf" not found). \\
	\ \ : \\
	-- (re)generating setup file ... \\
	\ \ :
}{chap-1-2-setup.eps}{Setup on Windows}
\medskip
セットアップ作業は以上です。

\medskip
\begin{small}
(注) \tt{devenv}が複数見つかった場合は次のようになりますので、
	適切な番号を選択してください。

\CmndLine{%
	-- checking devenv ... selection\_number: None \\
	found multiple "devenv" \\
	Select number (or '0' to try another one) \\
	\hspace{10pt}(1) Visual Studio Community 2017 (15.9.28307.222) \\
	\hspace{10pt}(2) Visual Studio Community 2019 (16.9.31129.286) \\
	\hspace{10pt}(0) try another one \\
	enter number:
}{chap-1-2-multiple-devenv.eps}{Found multiple devenv}
\end{small}

\subsubsection{再セットアップ}
\label{subsubsec:ReSetupOnWindows}

初めてのセットアップと同様、次のようにしてスクリプトを実行してください。

\CmndLine{%
	> chdir C:/Springhead/core/src \\
	> setup.bat
}{chap-1-2-re-setup.eps}{Re-Setup on Windows}

\medskip
必要な環境に変更がなければ

\CmndLine{%
	check result is ... \\
	-- no need to execute 'setup'. \\
	done
}{chap-1-2-no-need-to-re-setup.eps}{No need to Re-Setup}

となり、スクリプトは終了します。

何らかの変更がある場合、たとえば

\Vskip{-.3\baselineskip}
\begin{itemize}
  \item	使用するVisual Studioのバージョンを変更する場合
	(新しい Visual Studio をインストールした場合など)
  \item	使用するツールのバージョン/PATHを変更する場合
\end{itemize}

などの場合には

\CmndLine{%
	check result is ... \\
	-- setup is required (reason: "devenv" path differs, \Cont\\
	\hspace{60pt}"nmake" path differs). \\
	\\
	continue? [y/n]: 
}{chap-1-2-re-setup-needed.eps}{Re-Setup is needed}

などとなりますので、\tt{y}で答えてください。
必要な処理が実行され、セットアップファイルが更新されます。

\Important{%
	何らかの理由で上記の``\tt{continue? [y/n]:}''が表示されないとき、
	または強制的にセットアップファイルを再作成したいときには、
	\tt{setup}コマンドに\,\tt{-f}オプションを付けて実行してください。
}

\bigskip
% end 1.2.SetupOnWindows.tex
