% 1.5.EmbPython.tex
%	Last update: 2020/04/01 F.Kanehori
\newpage
\subsection{EmbPython}
\label{subsec:EmbPython}
\parindent=0pt

Python インタプリタに対する外部拡張モジュールを生成する 方法について説明します。

\medskip
\thinrule{\linewidth}

\noindent
\bf{Springhead アプリケーションにPythonインタプリタを組み込む場合}

\bigskip
\bf{Windowsの場合}
\begin{narrow}
	アプリケーションをビルドするとき、\QCMakeSettings{}のビルド条件
	\Path{SPR\_PROJS}に\Path{EmbPython}を加えてビルドしてください。
	ライブラリファイルが\Path{C:/Springhead/core/src/EmbPython}に
	次の名前で生成され、アプリケーションプログラムに組み込まれます。
 	(プラットフォームが 32 ビットの場合は x64 が x86 となります)。
	
	\medskip
	\begin{narrow}[20pt]
	\begin{tabular}{l@{\ \ ---\ \ }ll}\hline
	    & \multicolumn{1}{c}{Visual Studio 2015}
	    & \multicolumn{1}{c}{Visual Studio 2017} \\\hline
	    Debug構成 & \tt{EmbPython14.0x64D.lib} & \tt{EmbPython15.0x64D.lib} \\
	    Relase構成 & \tt{EmbPython14.0x64.lib} & \tt{EmbPython15.0x64.lib} \\
	    Trace構成 & \tt{EmbPython14.0x64T.lib} & \tt{EmbPython15.0x64T.lib} \\\hline
	\end{tabular}
	\end{narrow}

	\bigskip
	ビルド条件については\KQuote{\ref{subsec:PreparationForApp} ビルドの準備}
	を参照してください。

	また、次の\KQuote{unix の場合}と同様にすれば、
	上記のライブラリファイルだけを作成することもできます。
\end{narrow}

\medskip
\bf{unixの場合}
\begin{narrow}
	ディレクトリ\Path{C:/Springhead/core/src/EmbPython}に移動してください。

	次のようにcmakeコマンド(パラメータに注意)を実行してから
	makeコマンドを実行してください。

	\CmndLine{%
		> chdir C:/Springhead/core/src/EmbPython\\
		> cmake -B build -DSTANDALONE=1 [\it{generator}]\\
		> chdir build\\
		> make
	}{command-1-5-1.eps}{EmbPython-unix}

	\begin{narrow}[s]
		\it{generator}については
		\KQuote{\ref{subsec:CmakeLibrary} cmake} を参照してください。
	\end{narrow}

	\medskip
	ライブラリファイルは、
	ディレクトリ\Path{C:/Springhead/generated/lib}に次の名前で作成されます。
	(\tt{SHARED}構成は未だ実装されていません)

	\medskip
	\begin{narrow}[20pt]
	\begin{tabular}{l@{\ \ ---\ \ }l}\hline
		\tt{STATIC}構成 & \tt{libEmbPython.a}\\\hline
		\tt{SHARED}構成 & \tt{libEmbPython.so}\\\hline
	\end{tabular}
	\end{narrow}

	\bigskip
	アプリケーションをリンクするときは、
	このライブラリをSpringheadライブラリより前に指定してください。
\end{narrow}

\bigskip
\thinrule{\linewidth}

\noindent
\bf{Pythonインタプリタに対する外部拡張モジュール(Python DLL, pyd)を作成する場合}

\bigskip
\bf{Windowsの場合}
\begin{narrow}
	\begin{narrow}[s]
		%\Vskip{-.2\baselineskip}\thinrule{\linewidth}\\
		\thinrule{\linewidth}\\
		以下の作業は、\KQuote{\ref{subsec:CmakeLibrary}} cmake に従い
		\Path{C:/Springhead/core/src}においてcmakeが
		既に実行されていることを前提としています。
		さもないと、\\
		\hspace{10pt}\tt{\small{FileNotFoundError: [WinError 2]
		指定されたファイルが見つかりません。}}\hfill\\
		\hspace{20pt}\tt{\small{%
		 : 'C:/Springhead/core/src/build'}}\hfill\\
		というエラーが発生してビルドに失敗します。\\
		\Vskip{-1\baselineskip}\thinrule{\linewidth}\\
	\end{narrow}
	\Vskip{-1\baselineskip}
	ディレクトリ\Path{C:/Springhead/core/src/EmbPython}に移動してください。

	次のようにcmakeを実行します。

	\CmndLine{%
		> chdir C:/Springhead/core/embed/SprPythonDLL\\
		> mkdir build\\
		> cmake -B build [\it{generator}]\\
	}{command-1-5-2.eps}{SprPythonDLL}

	\begin{narrow}[s]
		\it{generator}については
		\KQuote{\ref{subsec:CmakeLibrary} cmake} を参照してください。
	\end{narrow}

	その後\Path{build}に移動し、\Path{SprPythonDLL.sln}を
	ビルドしてください。
	DLLファイルは\Path{C:/Springhead/generated/bin/\it{arch}}に
	次の名前で作成されます
	(\it{arch}は\Path{win64}または\Path{win32}です)。

	\medskip
	\begin{narrow}[20pt]
	\begin{tabular}{ccc}\hline
	    Debug構成 & Release構成 & Trace構成 \\\hline
	    \tt{SprD.py} & \tt{Spr.py} & \tt{SprT.py} \\\hline
	\end{tabular}
	\end{narrow}

\end{narrow}

\bigskip
\bf{unixの場合}
\begin{narrow}
	unix版は未だ実装されていません。
\end{narrow}

\medskip
\Important{%
	Windows Visual Studio用のソリューションファイルを用いてビルドする方法
	については
	\KQuote{\ref{subsec:DoNotUseCmake} CMakeを利用しない場合 (非推奨)}を
	ご覧ください。
}
\bigskip

% end: 1.5.EmbPython.tex
