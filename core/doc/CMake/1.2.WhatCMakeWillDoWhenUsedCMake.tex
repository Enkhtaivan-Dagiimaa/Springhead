% 1.2.WhatCMakeWillDoWhenUsedCMake.tex
%	Last update: 2019/07/05 F.Kanehori
\newpage
\subsection{CMakeを使用した場合}
\label{subsec:WhatCMakeWillDoWhenUsedCMake}

\cmake はビルドの自動化を図るためのツールであり、
ソースコードをコンパイルしたりデバッグの制御をしたりすることを目的とした
ツールではありません。
Visual Studioやmakeではなくconfigureに近いツールと考えていただいてよいでしょう。

\cmake の特徴の一つにout of source (out of place)でのビルドに対応している
ということがあります。
これはソースツリーの外側にビルドツリーを生成する機能で、
\begin{itemize}
  \item	複数のビルドツリーを作成することが可能である。
  \item	ビルドツリーが削除されてもソースツリーに影響が及ばない。
\end{itemize}
という特徴があります (Wikipediaより)。
ここでは\cmake をout of sourceで使用するものとして説明します。

\medskip
ソースツリーおよびビルドツリーは次のようになるでしょう。

\medskip
\begin{narrow}
    \begin{narrow}\begin{minipage}{\textwidth}
	\dirtree{%
		.1 \hspace{-10mm}.../Springhead/core/src/ \Anno{ソースツリー}.
		.2 CMakeLists.txt.
		.2 Base/ \Anno{(*1)}.
		.3 CMakeLists.txt.
		.3 Affine.cpp.
		.3 :.
		.2 Collision/ \Anno{(*2)}.
		.3 :.
		.2 :.
		.2 \build \Anno{ビルドツリー (他の場所でも構わない)}.
		.3 Springhead.sln.
		.3 Base/.
		.4 Base.vcxproj.
		.4 Base.dir/ \Anno{オブジェクトファイル(\tt{.obj})が置かれる}.
		.4 CMakeFiles/.
		.5 generate.stamp \Anno{スタンプファイル}.
		.3 Collision/.
		.3 :.
	}
	\medskip
  \end{minipage}\end{narrow}
\end{narrow}
\begin{center}図1 Springhead Library構成 \end{center}

\newpage
\begin{narrow}
    \begin{narrow}\begin{minipage}{\textwidth}
	\dirtree{%
		.1 \hspace{-10mm}.../Application/ \Anno{ソースツリー}.
		.2 CMakeLists.txt.
		.2 main.cpp.
		.2 Sub/.
		.3 CMakeLists.txt.
		.3 sub.cpp.
		.3 :.
		.2 Base/ \Anno{Spsringheadの(*1)を指すようにする}.
		.2 Collision/ \Anno{Spsringheadの(*2)を指すようにする}.
		.2 :.
		.2 \build \Anno{ビルドツリー (他の場所でも構わない)}.
		.3 Application.sln.
		.3 Base/.
		.4 Base.vcxproj.
		.4 Base.dir/ \Anno{オブジェクトファイル(\tt{.obj})が置かれる}.
		.4 CMakeFiles/.
		.5 generate.stamp.
		.3 Collision/.
		.3 :.
	}
	\medskip
  \end{minipage}\end{narrow}
\end{narrow}
\begin{center}図2 Application構成 \end{center}

\medskip
\noindent
ここで\Path{CMakeLists.txt}は\cmake を制御するためのパラメータファイルで、
メインディレクトリおよびサブディレクトリのそれぞれに作成します。

% end: 1.2.WhatCMakeWillDoWhenUsedCMake.tex
