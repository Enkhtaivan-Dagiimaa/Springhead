% 3.3.BuildApplication.tex
%	Last update: 2019/07/22 F.Kanehori
%newpage
\subsection{アプリケーションプログラムのビルド}
\label{subsec:BuildApplication}

\noindent
アプリケーションのビルドは従来と変わりがありませんが、
ソリューションに新しいターゲット
\tt{ALL\_BUILD}, \tt{RunSwig\_Clean}, \tt{sync}が追加されています。
ここでは、これらについて説明します。

\medskip
\noindent
\tt{ALL\_BUILD}
\begin{narrow}[20pt]
	これは\cmake が自動的に作成するターゲットでmake allに相当するものと
	されています。ただしVisual Studio上ではALL\_BUILDの依存関係の設定が不正確で、
	このターゲットをビルドしても正しい結果は得られないようです。
	{\bf{このターゲットは無視してください}}
\end{narrow}

\noindent
\tt{RunSwig\_Clean}
\begin{narrow}[20pt]
	従来の方法では、ターゲットRunSwigは\KQuote{メイクファイルプロジェクト}として
	作成されており、このターゲットには\KQuote{ビルド}, \KQuote{リビルド},
	\KQuote{クリーン}のそれぞれに対して別々に適切なコマンドを設定することが
	できました。
	しかし\cmake では、
	残念ながら\KQuote{メイクファイルプロジェクト}を作成することができず、
	全体を\KQuote{リビルド}しようとてもRunSwigだけは\KQuote{リビルド}されません。
	\begin{narrow}[s][15pt]
		これは、\cmake で作成されたRunSwigターゲットは
		カスタムビルドコマンドを実行するためのターゲットで、
		ここで設定されているコマンドは
		\KQuote{ビルド}時にしか実行されないためです。
	\end{narrow}
	したがって、
	{\bf{全体を(もしくはRunSwigを)\KQuote{リビルド}しようとするときは、
	その前に必ずRunSwig\_Cleanターゲットを\KQuote{ビルド}する必要があります。}}
	一手間増えますが、忘れないようにしてください。
\end{narrow}

\noindent
\tt{sync}
\begin{narrow}[20pt]
	これは\KQuote{\ref{subsec:CmakeApplication} cmakeの実行}で述べたとおり、
	プロジェクトファイルの整合性を保つために作られたものです。
	\SprLib のプロジェクトを一つでもビルドすれば
	このターゲットは必ず最初に実行されますから、
	このターゲットに対して何らかのアクションを起こす必要はないでしょう。
\end{narrow}


% end: 3.3.BuildApplication.tex
