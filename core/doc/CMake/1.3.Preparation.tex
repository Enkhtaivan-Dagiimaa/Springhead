% 1.2.Preparation.tex
%	Last update: 2020/02/18 F.Kanehori
%\newpage
\subsection{ビルドの準備}
\label{subsec:Preparation}
\parindent=0pt

ダウンロードが済んだら、ディレクトリ\SprTop{/core/src}に移動してください。

ライブラリのビルドに関連する配布ファイルは次のものです。

\begin{center}
\begin{tabular}{l@{\ \ ---\ \ }l}\hline
	\tt{\CMakeLists{.dist}} & ライブラリ生成用設定ファイル \\
	\tt{\CMakeSettings{.dist}} & ビルドパラメータ変更用ファイル \\
	\tt{\CMakeOpts{.dist}} & デフォルトビルドパラメータファイル \\
	\tt{\CMakeConf{.dist}} & 外部パッケージ・インストール先設定用ファイル \\\hline
\end{tabular}
\end{center}

\bigskip
セットアップを実行していない場合には、
配布されたファイル\QCMakeLists{.dist}を\QCMakeLists{}という名前でコピーしてください
(2.4c版 修正)。

\CmndLine{%
	> chdir C:/Springhead/core/src\\
	> \it{copy} \CMakeLists{.dist} \CMakeLists{}
}{command-1-3-a.eps}{CMakeLists.txt}

\bigskip
ビルド条件を変更する場合には、
配布されたファイル\Path{CMakeSettings.txt.dist}を\Path{CMakeSettings.txt}という名前で
コピーして、変数を適宜変更してください。

\CmndLine{%
	> \it{copy} \CMakeLists{.dist} \CMakeLists{}\\
	> \it{edit} \CMakeLists{}
}{command-1-3-b.eps}{CMakeSettings.txt}

\begin{itemize}
  \item	Visual Studioのバージョンを変更する場合: \\
	変数\tt{VS\_VERSION}を修正します(デフォルトは\tt{15.0}です)。
	ここで指定した値は、生成されるライブラリファイルの名前に組み込まれます。\\
	\bf{注意} ここで指定する値と生成されるソリューションファイルのバージョンとは
	関係がありあません
\end{itemize}

\medskip
配布されたビルド条件で問題なければ、これで準備は終了です。
\KQuote{\ref{subsec:Build} ビルド}へ進んでください。

\bigskip
独自にインストールしたパッケージ boost, glew, freeglut, gluiを使用する場合
およびライブラリファイルとヘッダファイルのインストール先を指定する場合には、
配布されたファイル\QCMakeConf{.dist}を\QCMakeConf{}という名前でコピーして
必要な編集をします。
編集の方法は\QCMakeConf{}に記述されています。

\def\somewhere{"C:/\it{somewhere}/\it{appropreate}"}
\CmndLine{%
	> \it{copy} \CMakeConf{.txt} \CMakeConf{}\\
	> \it{edit} \CMakeConf{}\\
}{command-1-3-c.eps}{CMakeConf.txt}

\bigskip
\bf{独自にインストールしたパッケージを使用するとき} (2.3版 改訂)

変数\tt{CMALE\_PREFIX\_PATH}にパッケージを探索するパスを設定します。
\def\cite#1{\hspace{10pt}\footnotesize{#1}}
\CmndLine{%
	\cite{set(CMAKE\_PREFIX\_PATH "C:/somewhre/appropreate")} \\
	\cite{\hspace{20pt}\#\hspace{40pt}{(\rm{use absolute path)}}}\\
	\cite{\hspace{20pt}\#\hspace{40pt}%
		{(\rm{multiple paths must be separated by `newline' or `semicolon')}}}\\
	\cite{)}\\
}{command-1-3-d.eps}{PackageImport}
\medskip
\begin{narrow}[s]
	実際に探索するパスは、\tt{<prefix> | <prefix>/(cmake|CMake) |}
	\tt{<prefix>/<name>* | <prefix>/<name>*/(cmake|CMake)}などです。
	ここで\tt{\,<prefix>\,}は変数に設定したパス、
	\tt{\,<name>\,}はパッケージ名を表します。
	詳細はcmakeのドキュメントを参照してください。
\end{narrow}

\bigskip
\bf{ライブラリファイルとヘッダファイルのインストール先を設定するとき} (2.3版 改訂)

\begin{narrow}[s]
	\Vskip{-.2\baselineskip}\thinrule{\linewidth}\\
	変数の設定はGUIからでも行なえます。
	ただし GUIでの設定より\Path{CMakeConf.txt}の設定の方が優先されます。
	また、GUIで\tt{CMAKE\_INSTALL\_PREFIX}を空に設定した場合、
	または \Path{CMakeConf.txt}で \tt{DO\_NOT\_GENERATE\_INSTALL\_TARGET}
	を指定した場合には、
	インストールを行なうためのターゲットルールは生成されません。
	なお、\Path{CMakeConf.txt}で\tt{CMAKE\_INSTALL\_PREFIX}に空文字列を
	設定した場合には、GUIの場合とは異なり、デフォルトのインストールパス
 	(Windowsでは\Path{C:/Program Files}、unixでは\Path{/usr/local})が
	使用されます。(2.4b版 追加)
	\Vskip{-.2\baselineskip}\thinrule{\linewidth}\\
	\Vskip{.3\baselineskip}
\end{narrow}

変数\tt{CMAKE\_INSTALL\_PREFIX}にインストール先を絶対パスで設定します。
この設定をすることにより、\tt{find\_package(Springhaed, REQUIRED)}として
\SprLib を簡単に導入することができるようになります。
\CmndLine{%
	\cite{set(CMAKE\_INSTALL\_PREFIX "C:/where/to/install")} \\
	\cite{\hspace{20pt}\#\hspace{40pt}{(\rm{use absolute path)}}}\\
}{command-1-3-e.eps}{PackageExport}
\medskip
実際にファイルがインストールされるディレクトリは
\Vskip{-.3\baselineskip}
\def\IP{\$\{CMAKE\_INSTALL\_PREFIX\}}
\begin{narrow}[s]
	\begin{tabular}{lcl}
		\tt{config files} & → & \tt{\IP/Springhead} \\
		\tt{header files} & → & \tt{\IP/Springhead/include} \\
		\tt{library file} & → & \tt{\IP/Springhead/lib} \\
	\end{tabular}
\end{narrow}
\Vskip{.2\baselineskip}
です。
これらのうち、ヘッダファイルとライブラリファイルのインストール先は
次の変数を設定することで変更が可能です。
絶対パスまたは\tt{\,\IP\,}からの相対パスで指定してください。
\Vskip{-.3\baselineskip}
\CmndLine{%
	\cite{set(SPR\_HEADERS\_INSTALL\_DIR "header files のインストール先")} \\
	\cite{set(SPR\_LIBRARY\_INSTALL\_DIR "library file のインストール先")} \\
}{command-1-3-f.eps}{OldInstall}

\bigskip
\small{\bf{[注意]}}

\begin{narrow}[s]
	次の変数は以前のバージョンとの互換のために残してありますが、
	将来的には廃止する予定です。
	なお、これらの変数を設定してもfind\_packageには対応しません。
	\CmndLine{%
		\cite{set(SPRINGHEAD\_INCLUDE\_PREFIX       "C:/somewhere/appropreate)}\\
		\cite{set(SPRINGHEAD\_LIBRARY\_DIR\_DEBUG   "C:/somewhere/appropreate)}\\
		\cite{set(SPRINGHEAD\_LIBRARY\_DIR\_RELEASE "C:/somewhere/appropreate)}
	}{command-1-3-g.eps}{OldInstall}
\end{narrow}

\bigskip
コンパイル及びリンクのオプションはファイル\QCMakeOpts{.dist}に設定されています。
これらのオプションを変更するときは、配布されたファイル\QCMakeOpts{.dist}を
\QCMakeOpts{}という名前でコピーして必要な編集をします。

\CmndLine{%
	> \it{copy} \CMakeOpts{.dist} \CMakeOpts{}\\
	> \it{edit} \CMakeOpts{}\\
	\hspace{20pt}:\\
}{command-1-3-h.eps}{CMakeOpts.txt}

\bigskip
以上で準備作業は終了です。

\medskip
\small{\bf{[参考]}}
\begin{narrow}[s][20pt]
	変数\it{variable}に値\it{value}を設定するには
	\tt{set(\it{variable} "\it{value}"}) とします。
	途中に空白やセミコロンを含まない文字列ならば引用符は省略できます。
	また、\tt{\$\{\it{variable}\}}とすると他の変数の値を、
	\tt{\$ENV\{\it{variable}\}}とすると環境変数の値を参照できます。
	文字`\tt{\#}'以降はコメントです。
\end{narrow}

\bigskip
\thinrule{\linewidth}

\bf{[注意]}

ダウンロード後最初のビルド時にswigのバイナリを生成しますが、
この時使用するプログラム\tt{devenv.exe}がパスに含まれていないと、
RunSwigに失敗して\SprLib が作成できません。
これはVisual Studioをデフォルト以外の場所にインストールしている場合などで起きる現象です。

この現象は、次のようにすると解消できます。

\begin{enumerate}
  \item	``Visual Studio用開発者コマンドプロンプト''を起動する
  \item	コマンド\Path{where devenv.exe}を実行してパスを調べる
  \item	上で調べたパスを環境変数\tt{PATH}に加える
  \item	Visual Studioを再起動する
\end{enumerate}

% end: 1.2.Preparation.tex
