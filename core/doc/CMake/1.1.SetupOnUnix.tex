% 1.1.SetupOnUnix.tex
%	Last update: 2021/05/13 F.Kanehori
%\newpage
\subsection{unixでのセットアップ}
\label{subsec:SetupOnUnix}
\parindent=0pt

\subsubsection{初めてのセットアップ}
\label{subsubsec:FirstSetupOnUnix}

ディレクトリ\SprTop{/core/src}に移動して、
スクリプト\tt{setup.sh}を実行します。

\CmndLine{%
	> chdir .../Springhead/core/src \\
	> ./setup.sh \\
	found python (Version 3.4.2) \\
	setup file ("setup.conf") not found. \\
	\\
	currently available binaries are ... \\
	-- checking python ... found (version: 3.4.2) \\
	-- checking gcc ... found (version: 7.5.0) \\
	-- checking swig ... found (version: 2.0.4) \\
	-- checking cmake ... found (version: 3.18.0) \\
	-- checking gmake ... found (version: 4.1) \\
	-- checking nkf ... found (version: 2.1.4 (2015-12-12)) \\
	\\
	check result is ... \\
	-- setup is required (reason: setup file "setup.conf" not found, \\
	"CMakeLists.txt" not found). \\
	\\
	continue? [y/n]: 
}{chap-1-1-setup.eps}{Setup}
\medskip
ここで\tt{y}と答えればセットアップファイルが作成されます。
セットアップ作業はこれで終わりです。

\subsubsection{再セットアップ}
\label{subsubsec:ReSetupOnUnix}

初めてのセットアップと同様、次のようにしてスクリプトを実行してください。

\CmndLine{%
	> chdir .../Springhead/core/src \\
	> ./setup.sh
}{chap-1-1-re-setup.eps}{Re-setup}

\medskip
必要な環境に変更がなければ

\CmndLine{%
	\ \ : \\
	check result is ... \\
	-- no need to execute 'setup'. \\
	done
}{chap-1-1-no-need-to-re-setup.eps}{No need to re-setup}

となり、スクリプトは終了します。

何らかの変更がある場合、たとえば

\Vskip{-.3\baselineskip}
\begin{itemize}
  \item	使用するツールのバージョン/PATHを変更する場合
  \item	拡張版swigの再ビルドが必要となったとき
\end{itemize}

などの場合には

\CmndLine{%
	\ \ : \\
	check result is ... \\
	-- setup is required (reason: "swig" need to be rebuilt). \\
	\\
	continue? [y/n]: 
}{chap-1-1-re-setup-needed.eps}{Re-setup is needed}

などとなりますので、\tt{y}で答えてください。
必要な処理が実行され、セットアップファイルが更新されます。

\Important{%
	何らかの理由で上記の``\tt{continue? [y/n]:}''が表示されないとき、
	または強制的にセットアップファイルを再作成したいときには、
	\tt{setup}コマンドに\,\tt{`-f'}オプションを付けて実行してください。
}

\bigskip
% end 1.1.SetupOnUnix.tex
