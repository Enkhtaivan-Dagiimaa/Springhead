% 1.0.WhatCMakeWillDo.tex
%	Last update: 2019/07/24 F.Kanehori
\newpage
\section{CMakeとは何をするものか}
\label{sec:WhatCMakeWillDo}

\def\Anno#1{\rm{\footnotesize{ \CDots\  #1}}}
\ifLwarp
\def\SolutionFile{ソリューションファイル$(\dagger 1)$}
\def\ProjectFile{プロジェクトファイル$(\dagger 2)$}
\else
\def\SolutionFile{\hbox{ソリューションファイル{\small{$^{(\dagger 1)}$}}}}
\def\ProjectFile{\hbox{プロジェクトファイル{\small{$^{(\dagger 2)}$}}}}
\fi
\def\cmake{\tt{cmake}}
\def\make{\tt{make}}
\def\SprLib{Springheadライブラリ}
\def\SprProj{Springheadプロジェクト}
\def\VS{Visual Studio}

\def\SprTop#1{\Path{C:/Springhead{#1}}}
\def\AppTop#1{\Path{C:/Develop/Application{#1}}}
\def\build{{\it{build\/}}}

\def\CMakeLists#1{\Path{CMakeLists.txt{#1}}}
\def\CMakeOpts#1{\Path{CMakeOpts.txt{#1}}}
\def\CMakeConf#1{\Path{CMakeConf.txt{#1}}}
\def\CMakeTopdir#1{\Path{CMakeTopdir.txt{#1}}}

\def\thinrule#1{\makebox[#1][l]{\vrule width #1 height 0.1pt}}

\ifLwarp
\def\UpKQs{``}
\def\UpKQe{''}
\def\KQuote#1{``#1''}
\fi

\noindent
最初にCmakeとは何をするものかについて簡単に説明をします。
Springheadの開発に携わらない方はこの章の内容は不要です。
\UpKQs \ref{sec:ForNonDevelopper} Springheadをライブラリとして利用する方へ\UpKQe
をお読みください。
また、Cmakeについて理解をされている方は
\UpKQs \ref{subsec:Problems} CMakeを使用した場合の問題点\UpKQe、
\UpKQs \ref{subsec:Solution} 対処法\UpKQe および
\UpKQs \ref{sec:ForDevelopper} Springheadを開発する方へ\UpKQe をお読みください。

% end: 1.0.WhatCMakeWillDo.tex
